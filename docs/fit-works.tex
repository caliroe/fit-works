\documentclass[conference]{IEEEtran}
\IEEEoverridecommandlockouts
% The preceding line is only needed to identify funding in the first footnote. If that is unneeded, please comment it out.
\usepackage{cite}
\usepackage{amsmath,amssymb,amsfonts}
\usepackage{algorithmic}
\usepackage{graphicx}
\usepackage{textcomp}
\usepackage{xcolor}

\begin{document}

\title{Fit Works}

\author{\IEEEauthorblockN{ Pietro Nazar }
\IEEEauthorblockA{\textit{Graduando em Ciência da Computação} \\
\textit{Centro Universitário IESB}\\
Brasília, DF\\}

\and
\IEEEauthorblockN{\textsuperscript\\{\\} Gabriel Assunção de Oliveira}
\IEEEauthorblockA{\textit{Graduando em Engenharia da Computação} \\
\textit{Centro Universitário IESB}\\
Brasília, DF\\\\gabriel.assuncao@iesb.edu.br}

\and
\IEEEauthorblockN{\textsuperscript\\{\\} Daniella Téu}
\IEEEauthorblockA{\textit{Graduando em Ciência da Computação} \\
\textit{Centro Universitário IESB}\\
Brasília, DF\\}

}

\maketitle


\begin{resumo}
\textbf{{\textit{Resumo}}--Este artigo irá abordar o desenvolvimento de nossa aplicação que tem por objetivo auxiliar o usuário a se exercitar com o auxílio de instruções personalizadas do aplicativo, cuja informações coletadas no início serão repassadas a um profissional que irá definir os exercícios e alimentar uma inteligência artificial }
\end{resumo}


\begin{IEEEkeywords}
Palavras-chave: Fake news, Notícias falsas, Inteligência artificial, Machine learning.
\end{IEEEkeywords}

\section{Introdução}

A pandemia veio e fez com que medidas que alteram a rotina de todo mundo fossem tomadas, a mais marcante foi o isolamento em casa. Pessoas que antes saiam de casa, tinham atividades que faziam com que se movimentasse e espaços públicos para realização de atividades, se encontram “presas” em casa, tendo que realizar todas as suas atividades de trabalho e lazer dentro de um espaço delimitado. O preço disso foi um aumento do sedentarismo. Os ossos, músculos e articulações, por causa da baixa mobilidade, perderam resistência. Segundo o Google Trends, o termo “dor nas costas” teve crescimento de 76% desde o início do novo coronavírus no país. Quase seis em cada dez brasileiros (57,25%) estavam com sobrepeso em 2021. Uma vida sedentária leva a pessoa a um maior risco de ter um acidente vascular cerebral (AVC), insuficiência renal crônica, cardiopatia e até problemas de visão. O sedentarismo também é muito associado à diabetes e à obesidade, aumentando os riscos de doenças cardiovasculares. Diante desses perigos, muitas pessoas foram buscar na tecnologia, em especial nos aplicativos, opções de sair dessa condição.

\section{Contexto}

Com todas as consequências causadas pelo isolamento, veio a necessidade de se realizar exercícios em casa para uma vida saudável. Mas como fazer?

Muitos profissionais e grandes marcas utilizam aplicativos, com abordagens diferentes. O foco das empresas (como Nike e Adidas, dois grandes aplicativos gratuitos disponibilizados em português) era num conteúdo “faça você mesmo” com exercícios simples, com vídeos e figuras orientando a melhor maneira de realizar, com poucos ou nenhum equipamento e utilizando objetos e o espaço da casa, baseados em informações simples que os usuários poderiam facilmente informar ao começar a utilizar o aplicativo. Já os profissionais se utilizam de sua especialidade para oferecer um acompanhamento mais específico às necessidades do usuário, se utilizando de dados mais completos e personalizando os exercícios e atividades que poderiam ser realizadas dentro de casa.


\section{Problema}
Com base no artigo “efetividade de aplicativos móveis para mudanças comportamentais em saúde: revisão sistemática”, vimos que os aplicativos são efetivos na construção de uma rotina e hábitos saudáveis do usuário (levando em conta contextos sociais/políticos específicos), entretanto sua precisão fica comprometida pela ineficiência de juntar a agilidade de soluções tecnológicas com o conhecimento de profissionais da área.


\section{Objetivo Geral}\label{AA}
O objetivo é criar uma aplicação que inicialmente tenha uma input com os dados básicos do usuário e em seguida demonstra exercícios básicos e simples que podem ser implementados no dia a dia dele. Logo após isso, um profissional o acompanhará pelo aplicativo e irá definir as melhores rotinas e atividades para o usuário com vista de alcançar o objetivo de próprio. Em paralelo a tudo isso, um algoritmo de Inteligência Artificial irá aprender com o profissional para que possa entregar cada vez melhor os exercícios do usuário logo após o cadastro realizado.

\section{Objetivo Específico}

* Tela inicial: Login ou cadastro (aluno e professor)
* Professor: Acesso administrador e ao perfil dos clientes
* Professor: Tela home, na qual é possível selecionar alunos, incluir treinos, planos de exercícios e fazer as alterações desejadas
* Professor: Alunos, tela na qual o profissional seleciona o aluno para acompanhamento
* Professor: Agenda, para controle e marcação de horários
* Professor: Financeiro, para acompanhamento de pagamentos.
* Aluno: Primeiro acesso, campo para informações básicas (nome, telefone, data de nascimento e endereço), informações de saúde acessíveis a qualquer leigo (peso e altura)
* Aluno: Numa lógica básica e genérica (posteriormente alimentada por IA com tentativa e erro validada por um profissional) é ofertada ao usuário treinos iniciais antes do contato de um profissional
* Aluno: Tela home, na qual o usuário pode navegar entre as opções do app (Perfil, treino, progresso, financeiro, avaliação, arquivos)
* Aluno: Progresso, acompanhamento com um profissional do progresso alcançado com as práticas do exercício
* Aluno: Financeiro, para consultas de extrato de pagamento
* Aluno: Avaliação, para acompanhamento de metas, contatos e feedbacks
* Aluno: Arquivos, para material (PDF, DOC, etc) disponibilizados pelo profissional ao aluno.


\section{Referencial Teórico}

\begin{itemize}
\item [1] Martins, André. Pandemia eleva índice de sedentarismo entre a população brasileira. Disponível em: (https://exame.com/bussola/pandemia-eleva-indice-de-sedentarismo-entre-a-populacao-brasileira/). Acesso em: 07/04/2022.

\end{itemize}

\begin{itemize}
\item [2] Welter Ritter ,Eduardo; Rigo, Sandro José. FITDATA: Um sistema para monitoramento de atividade física baseado em dispositivos móveis. Pós-Graduação em Computação Aplicada.  XII Brazilian Symposium on Information Systems, Florianópolis, SC, May 17-20, 2016.
\end{itemize}

\begin{itemize}
\item [3] Paula TR, Menezes AP, Guedes NG, Silva VM, Cardoso MVLML, Ramos ES. Effectiveness of mobile applications for behavioral changes in
health: a systematic review (Efetividade de aplicativos móveis para mudanças comportamentais em saúde: revisão sistemática). 
\newline
\end{itemize}


\section{Trabalhos Correlatos}

Com base no artigo “Efetividade de aplicativos móveis para mudanças comportamentais em saúde: revisão sistemática", queremos melhorar a precisão dos app de exercício de maneira que a IA e o instrutor humano possam trabalhar em conjunto, aumentando o grau de efetividade.

\end{document}